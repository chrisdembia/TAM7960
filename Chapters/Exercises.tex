\chapter{Exercises} %(notes page 67-71)
\label{sec:Exercises}

The following exercises were given as homework problems for the Spring 2010 offering of TAM7960 at Cornell University.

\catcode`\^^M=10      %  Makes blank lines meaningless to TeX,

%%%%%%%%%%%%%%%%%%%%%%%
\section{Cost of Transport}

\begin{enumerate}
\item A natural measure of energy use is ``The specific energetic
cost of transport",

\begin{equation}                  
COT \equiv \frac{[\textrm{Energy used}]}{[\textrm{weight}]\cdot[\textrm{distance}]}
\end{equation}

Note that weight is a force so $COT$ is a
dimensionless number.

Estimate, using whatever numbers you know or look up, the
$COT$ for the following.

\begin{itemize}
\item  a person walking
\item a person biking
\item a car (full weight)
\item a car (only counting weight of passengers)
\item a freight truck
\item a freight train
\item a passenger train (only counting weight of passengers)
\item any animal; walking, running or flying.
\end{itemize}

Make your assumptions clear and clearly identify any sources use.  It is best to have two
different estimates, based on different types of data.

\item What would be a good measure of locomotion speed for 
comparing different animals or robots?  Justify your answer
as best you can.  If you have more than one candidate, 
compare them. Use your own reasoning rather than finding metrics provided in the literature, etc.
\end{enumerate}

\newpage
%%%%%%%%%%%%%%%%%%%%%%%
\section{Motors}

\begin{center}
\begin{figure}[h]
\centering
\subfloat[worm drive.]{\includegraphics[scale =.7]{Figures/wormgear}}
\subfloat[wedge.]{\includegraphics[scale =1]{Figures/wedge}}
\end{figure}
\end{center}

\begin{enumerate}

\item  An ideal passive transmission conserves energy.
One example of a transmission is a screw-drive or ``worm drive''.   A model for the basic
mechanics of a worm drive is the wedge.

\begin{enumerate}

\item  For simplicity, assume that all sliding is frictionless and that
the parts have no mass or weight.  Think of A as the input and B as
the output. 

\begin{enumerate}

\item   Find $F_B$ in terms of $\theta$ and $F_A$.

\item From geometry (kinematics), find the sideways motion
of B in terms of $\theta$ and the motion of A.

\item Note that the results above are consistent with energy conservation.

\item Simplify all of the expressions above using a small angle approximation
($\theta \ll 1, \sin \theta \approx \theta$, \etc.)

\end{enumerate}

\item Now include friction, but just on the surface AB between the parts.  Assume a friction coefficient $\mu$ or friction angle $\phi$ with $\tan\phi = \mu$.  Assume that the
friction against the walls is still zero.  For the worm drive this is like assuming
that the shafts have good bearings but that all the friction is on the screw
surface.  Assume the wedge is traveling down. 

\begin{enumerate}
\item  Find $F_B$ in terms of $\theta$,  $F_A$ and $\mu$  (or $\phi$).
\item  Note that the kinematics is unchanged by the friction.
\item  What fraction of the power is lost?
\item  Now assume the drive is running backwards, with the wedge A
 being pushed up.  Now what fraction of the power is lost?
 
 \item Assume small angles.  Show that if $\mu$ (or $\phi$) and $\theta$ are
 such that 50\% of the power is lost when A is pushed down, 100\% is lost when B is pushed in.
 
\end{enumerate}
\end{enumerate}

\par

\item  The most common idealization of a motor includes two energy dissipation terms:
$c\omega$ in damping and $IR$ in electrical resistance.

\begin{align}
T_{m} &=  k I_{m}  - c \omega & \textrm{Torque equation}\label{firstmotoreq}\\
V_{m} &= k \omega_{m} + I_{m} R + \dot I_{m} L & \textrm{Voltage equation} \label{secondmotoreq}
\end{align}


In these equations\par
\begin{center}
  \parbox{4 in}{
\begin{description}
\item[$T_m = $] the torque the motor exerts on what it's connected to,
\item[$\omega = $] the motor angular velocity, 
\item[$I_{m} = $] the electric current through the motor, 
\item[$V_{m} = $] the voltage across the motor,
\item[$k = $] the `motor constant' or
`torque constant',  
\item[$R = $] the internal electrical resistance of the motor, and
\item[$c = $] the viscous damping constant, which also includes electrodynamic
drag,
\item[$L = $] the motor inductance which is not important for steady state operation with a DC power supply.
\end{description}
}

\end{center}

\par
From these we calculate other various quantities, for example the electric power in  and mechanical power out are 

\begin{equation}
P_{i} = V_{m} I_{m}  \centerand P_{o} = T_{m} \omega_{m}
\end{equation}


Now imagine that we also have access to a frictionless gear box with gear ratio $G$ that
changes the torque and angular velocity output to

\begin{equation}
T_{o} = G T_{m} \centerand \omega_{o} = \omega_{m}/G
\end{equation}

and also a lossless transformer with turns ratio $D$ that can 'transform' the battery
voltage to

\begin{equation}
V_{m}  =   D V_{in}  \centerand   I_{m} = I_{in}/D.
\end{equation}

You are given a motor with a fixed $k$, $c$, and $R$, and a battery with fixed $V_{in}$. Additionally, the output angular velocity ($\omega_o$) and motor power power $P_o$ are specified.

\begin{enumerate}


\item Find  $G$ and $D$ to maximize the efficiency of the system. 

\item  What is the maximized efficiency?

\item   Do the calculations above for a real motor.  The Faulhaber
4490-024B has a nominal voltage of  24 V.  This means you should use
24 V to infer the other motor constants.  The motor resistance is $R = 0.237$ $\mathrm{\Omega}$.  The  `no load' speed is 9550 rpm; this means that at 24V that $\omega =
9550 \times 2\pi/ 60\s$ when $T_m = 0$.  The stall torque is $2.406\Nm$. 
The no-load current is $0.554$ Amps (this is the current when the 24V is applied but no torque).  The torque constant is $k =0.02383 \Nm/ $Amp.

\begin{enumerate}
\item  Find the damping constant $c$.

\item  Assume that a mechanical power of  $P_{o} = 20 \watts$ is desired
at $\omega_{o} = 1 \rad/\s$ using a $24 V$ battery.  Find the best gear ratio and transformer ratio to minimize the power in $P_{i}$.  What is that minimized power?

\end{enumerate}
\end{enumerate}
\end{enumerate}

\newpage
%%%%%%%%%%%%%%%%%%%%%%%
\section{Hopping Frog}

\begin{enumerate}

\item \textbf{Reading.} Read Ruina 2005 \cite{ruina05}. Also read Rashevsky 1948 \cite{rashevsky48} if you are interested.

\item \textbf{The cost of frog jumping.}  A frog makes a sequence of
parabolic jumps, stopping and jumping again each time.  Neglect the
waiting time between jumps.

Here are some possibly relevant variables.\par
\begin{center}
  \parbox{4 in}{
\begin{itemize} 
\item[$d= $] step length (flight distance) 
\item[$\phi= $] jump angle, measured from the positive $x$ axis
\item[$t_f= $] time of flight
\item[$ v_{x0}= $] $x$ component of jump velocity = $v_{0} \cos{\phi}$
\item[$ v_{y0}= $] $y$ component of jump velocity = $v_ {0} \sin{\phi}$
\item[$v_{0} = $] launch speed $=\sqrt{v_{x0}^2+v_{x0}^2}$
\item[$v= $] average forward speed = $v_x = d/t_f$
\item[$h = $] height of flight
\item[$g= $] downwards gravity constant
\item[$m= $] mass of the frog
\item[$b = $] metabolic cost multiplyer $E_{m} = b|W_{neg}|$
\end{itemize}
}
\end{center}

\begin{enumerate}
\item In terms of some of the variables above determine the specific cost of transport.

\begin{equation}                  
COT \equiv \frac{[\textrm{Energy cost}]}{[\textrm{weight}]\cdot[\textrm{distance}]}
\end{equation}

Express this $COT$ at least two different ways, using two different minimal sets
of variables.

\item  For a given $m$, $g$, and $d$ find other variables to minimize $COT$. Find
the resulting $v$ and $COT$ in terms of $m$, $g$, and $d$. Plug in actual numbers with
$g= 10\mperss$ and $d=1\m$.

\item For a given $m,g,v$ find other variables to minimize $COT$. 

\item What interesting observations can you make about the calculations
above?

\end{enumerate}
\end{enumerate}

\newpage
%%%%%%%%%%%%%%%%%%%%%%%
\section{The Collisional Model}

\begin{enumerate}
\item  Equations 15 and 16 in Ruina 2005 \cite{ruina05} are based on a single metabolic inefficiency $b$. Rederive equations 15 and 16 using two cost coefficients $b_1$ and $b_2$, for negative and positive work respectively. The final formulas will be more complicated.

\item Use two cost coefficients $b_{1}$ and $b_{2}$ to replace the formulas for curves i and ii in Figure 5 of \cite{ruina05}. You need not use the formulas from problem (1) above, but can rederive them as needed for this special case.

\item  On the course motor page  there is a design sheet by David Palombo of Aveox (seller
of motors).  He says  ``The motor is at it's peak efficiency when the iron loss equals the copper loss.''  The `copper loss'  is $I^{2}R/2$.  The iron loss is, presumably, $c\omega^{2}/2$.

\begin{enumerate}

\item Check this result using specific simple motor parameters (use $k=10, R=1, c=1$ in consistent units). Also, create a set of multiple-line or vertically stacked plots that show, for a fixed $V$ (use 10), the quantities $P_{in}, P_{out}$, $I^{2}R/2$, and $c\omega^2/2$ as functions of $\omega$.  A vertical line should mark the point of maximum efficiency.  You can
use Matlab code, computer algebra or other such tools.

\item Use the numerical answer above to definitively answer the
questions: Is the result exactly true for our standard motor model? Is it approximately true?  Is it true in some limiting cases?

\end{enumerate}
\end{enumerate}

\newpage
%%%%%%%%%%%%%%%%%%%%%%%
\section{The Rimless Wheel}

This whole exercise is about the rimless wheel. The basic parameters are given here.
Where numerical values are needed, assume consistent units and use the numbers given in parentheses.

\parindent = 30 pt
\begin{description}
\item[$ m=$] mass of rimless wheel (1).
\item[$ I=I^{G}=$] moment of inertia about the center of the wheel, G (1/2).
\item[$ N=$] number of spokes (8).
\item[$ \phi=$] angle between spokes = $2\pi/N$.
\item[$ \ell =$] length of spokes (1).
\item[$ \gamma =$] slope of ground, down is to the left (0.1 radians).
\item[$ g=$]  gravity constant (1).
\end{description}

The basic variables are:
\begin{description}
\item[$  \theta_n=$] angle of stance leg $n$ relative to a line normal to the ground, measured CCW.
\item[$  \omega=\dot\theta=$] angular velocity of wheel, measured CCW.
\item[$ (\dots)_{n}^{+}=$] $(\dots)$ just after collision $n$.
\item[$ (\dots)_{n}^{-}=$] $(\dots)$ just before collision $n$.
\item[$f(\omega_{n}^{+})  = \omega_{n+1}^{+} = $] poincare map/return map/stride function.
\item[$ (\dots)_{*}^{-} =$] steady state value of $(\dots)$ if the motion is periodic.
\end{description}

\begin{enumerate}
\item Use energy conservation to find $\omega_{n+1}^-$ from $\omega_{n}^{+}$
\item Use numerical integration of the non-linear ODEs, with the appropriate
end condition, to verify  your answer to the previous problem for a range of
initial conditions.  

\item Plot $f$ along with the line $y=x$ using the same scale for both
axes.  Use your numerical evaluation to obtain the motion between collisions rather than using the energy conservation result above.

\item Find $\omega_{*}^{+}$ to at least 10 digits accuracy with a root finding method on your numerical evaluation of $f$.  Find it again analytically using the energy conservation result. Compare the solutions.

\item Start with the initial condition that $\omega_{0}^{+} = 1.5 \omega_{*}^{+}$.
Calculate $\omega_{n}^{+}$ for $n = 1 \dots 20$.

\item  Define a dimensionless quantity
\begin{equation}
r_{n} = \frac{ \omega_{n}^{+} -\omega_{*}^{+}}{ \omega_{n-1}^{+} -\omega_{*}^{+}}
\end{equation}
Plot $r_{n}$ vs $n$ and show that it tends to a constant.  Find that constant as best
you can. Describe the significance of $r_{n}$.

\item Find an analytic formula for the limit $n\rightarrow\infty$ of $r_n$, evaluate the result with the given numbers, and compare it to your numerical result above.


\end{enumerate}

\newpage
%%%%%%%%%%%%%%%%%%%%
\section{SLIP}

In this assignment, you will develop a computational model for spring mass (SLIP) running.
Here are the parameters of the model.  In parentheses are values you should use,
in consistent units, for numerical work.

\begin{description}
\item{$m=$} mass of running human (100).
\item{$ \theta_{c}=$} touchdown angle (variable).
\item{$\ell_{0} =$} uncompressed leg length (1).
\item{$ k=$}  spring constant (10,000; compression under gravity is 1/10 of leg length).
\item{$ g=$}  gravity constant (10).
\end{description}

The basic dynamic variables, and things you can measure, are:
\begin{description}
\item{$ x,y,\dot{x}, \dot{y}=$} position and velocity of COM
\item{$d=$} step length
\item{$ \bar{v} =$} average forward speed
\item{$h=$} height at top of periodic trajectory ($=y_{max}$)
\item{$c_{duty}$} fraction of time the foot is on the ground
\end{description}

Develop the model using section 3 from Figure \ref{fig:SLIPSections}, the peak of the flight.
Thus the section variables are $h$ and $\dot{x}$ with $\dot{y} = 0$.

\begin{enumerate}

\item Find a periodic motion using the computer recipe. That is, the fixed values $h$ and $\dot{x}$ at section 3 for each period.

\item Find a \textit{stable} periodic motion.

\end{enumerate}

Graph and animate the trajectories of the locomotive, showing the stance leg
during stance.  The activities of graphing and animation may also be useful for debugging purposes,
even before you have solutions to the problems above.

% THIS INPUTS MATLAB CODE:
%\newpage
%%%%%%%%%%%%%%%%%%%%%%%
%\section{MATLAB: SLIP Simulation}
%
%\lstinputlisting{SLIP_Simulation/Raibert_Control.m}
