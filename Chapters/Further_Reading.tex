\chapter{Further Reading}
\label{sec:FurtherReading}

As is evidenced by the nature of this text, work in the area of Legged Locomotion is varied. Further reading is organized along logical subfields, with some commentary on the tastes of the researchers or nature of the texts listed.

\subsubsection*{Dynamics and Mechanics}
\begin{itemize}
\item McGeer
\item Ruina: passive, simple models.
\item Collins
\end{itemize}

\subsubsection*{Muscles, Motors, and Energy}
\begin{itemize}
\item McMahon \cite{mcmahon84}
\item MacNeil Alexander
\end{itemize}

\subsubsection*{Experiments on humans and animals}
\begin{itemize}
\item Hoyt \cite{hoyt81}
\item Donelan Kuo \cite{donelan01}
\item Molen \cite{molen72b}
\item Muybridge (featured work at the NY library)
\end{itemize}

\subsubsection*{Robotics and control}
\begin{itemize}
\item Raibert has performed work on...
\item Ogata \cite{ogata04} is a standard system dynamics textbook that introduces the basics of feedback control.
\end{itemize}

\subsubsection*{Application of Locomotion Research}
\begin{itemize}
\item Hugh Herr prosthetics
\item Boston Dynamics \\ \\Example of text for further reading section entries: Perhaps the most well-known product of locomotion research at Boston Dynamics is the BigDog robot, a hydraulically-powered quadruped battlefield assistant. Boston Dynamics has also researched a bipedal version of this robot, called PetMan. Look them up on Youtube, they're awesome!
\end{itemize}

\subsubsection*{Locomotion-related Academic Progeny of Professor Ruina}
\begin{itemize}
\item Collins
\item Bertram
\item Coleman
\item Srivasian
\end{itemize}

[maybe for now this can just be a list of things we want to include here formally?. Prof. Ruina would be able to write up this section in a jiffy, probably]